\documentclass[a4paper]{extarticle}
\usepackage[utf8]{inputenc}
\usepackage[italian]{babel}
\selectlanguage{italian}
\usepackage[table]{xcolor}
\usepackage{xcolor}
\usepackage{circuitikz}
\usetikzlibrary{positioning, circuits.logic.US}
\usetikzlibrary{shapes.geometric, arrows}
\usetikzlibrary {shapes.gates.logic.US, shapes.gates.logic.IEC, calc}
\tikzset {branch/.style={fill, shape = circle, minimum size = 3pt, inner sep = 0pt}}
\usetikzlibrary{matrix,calc}
\usepackage{multirow}
\usepackage{float}
\usepackage{geometry}
\usepackage{tabularx}
\usepackage{pgf-pie}
\usepackage{tikz}
\usepackage{amsmath}
\usepackage{amssymb}
\usepackage{color, soul}
\usepackage{fancyhdr}
\usepackage{graphicx}
\usepackage{subfig}
\graphicspath{ {./img/} }
\newtheorem{theorem}{Teorema}[section]
\newtheorem{corollary}{Corollario}[theorem]
\newtheorem{lemma}[theorem]{Lemma}

% Specifiche
\geometry{
 a4paper,
 top=20mm,
 left=30mm,
 right=30mm,
 bottom=30mm
}

\nocite{*}

\pagestyle{fancy}
\fancyhf{}
\fancyhead[LO]{\nouppercase{\leftmark}}
\fancyfoot[CE, CO]{\thepage}
\addtolength{\headheight}{1em}
\addtolength{\footskip}{-0.5em}

\newcommand{\quotes}[1]{``#1''}
\renewcommand\tabularxcolumn[1]{>{\vspace{\fill}}m{#1}<{\vspace{\fill}}}
\renewcommand\arraystretch{}
\newcolumntype{P}{>{\centering\arraybackslash}X}

\title{\textbf{Seconda prova - A.A. 2021/22\\}}
\author{Cognome e Nome : Piccin Enrico}
\date{Matricola: IN0501089}

\begin{document}

\vspace{-10mm}
\maketitle

%\tableofcontents
%\newpage
\noindent
\section{Esercizio}
Si descriva il funzionamento attraverso la tavola di Huffman di un dispositivo asincrono dotato di tre segnali di controllo $CSR$ e di un'uscita $Z$. Si desidera che l'uscita si ponga allo stato alto quando il segnale $C$ è ALTO e viene rilevato un FRONTE di SALITA sul segnale $S$, mentre ritorni allo stato basso quando $C$ è ALTO e viene rilevato un FRONTE di SALITA sul segnale $R$. Se $C$ è allo stato BASSO il sistema mantiene l'uscita stabile indipendentemente dall'alternanza di segnali su $S$ ed $R$. La tavola di Huffman sia scritta usando una rappresentazione ordinata come quella qui sotto riportata

\vspace{1em}
\noindent
\begin{table}[H]
    \rowcolors{1}{white}{white}
\setlength{\tabcolsep}{4pt}
\renewcommand{\arraystretch}{1.2}
\centering
\begin{tabular}{c|cccc|cccc|}
    \hline
    $C$ & $0$ & $0$ & $0$ & $0$ & $1$ & $1$ & $1$ & $1$\\
    $SR$ & $00$ & $01$ & $11$ & $10$ & $00$ & $01$ & $11$ & $10$\\
    \hline
\end{tabular}
\end{table}

\vspace{1em}
\noindent
Inoltre per questa macchina si proponga una codifica che minimizzi le variabili di stato ma che sia priva di corse critiche.

\vspace{2em}
\noindent
La tavola di Huffman che descrive il funzionamento del circuito sequenziale asincrono richiesto dal problema è la seguente

\vspace{1em}
\noindent
\begin{table}[H]
    \rowcolors{1}{white}{white}
\setlength{\tabcolsep}{4pt}
\renewcommand{\arraystretch}{1.2}
\centering
\begin{tabular}{c|cccc|cccc|c}
    \hline
    $C$ & $0$ & $0$ & $0$ & $0$ & $1$ & $1$ & $1$ & $1$ & $Z$\\
    $SR$ & $00$ & $01$ & $11$ & $10$ & $00$ & $01$ & $11$ & $10$ & \\
    \hline
    $A$ & $A$ & $A$ & $A$ & $A$ & \cellcolor{orange!25} $A$ & \cellcolor{orange!25} $A$ & $D$ & $D$ & $0$\\
    $B$ & $B$ & $B$ & $B$ & $B$ & $A$ & $A$ & \cellcolor{orange!25} $B$ & \cellcolor{orange!25} $B$ & $0$\\
    $C$ & $C$ & $C$ & $C$ & $C$ & \cellcolor{orange!25} $C$ & $B$ & $B$ & \cellcolor{orange!25} $C$ & $1$\\ 
    $D$ & $D$ & $D$ & $D$ & $D$ & $C$ & \cellcolor{orange!25} $D$ & \cellcolor{orange!25} $D$ & $C$ & $1$\\
    \hline
\end{tabular}
\end{table}

\vspace{1em}
\noindent
Tramite l'adozione di transizioni multiple all'interno della tavola di Huffman, è stata anche semplificata la codifica dei degli stati, minimizzando le variabili ed eliminando eventuali corse critiche:

\vspace{1em}
\noindent
\begin{table}[H]
    \rowcolors{1}{white}{white}
\setlength{\tabcolsep}{4pt}
\renewcommand{\arraystretch}{1.2}
\centering
\begin{tabular}{c|c|c|}
    \hline
    $y_1$ & $0$ & $1$\\
    $y_2$ &     &    \\
    \hline
    $0$ & $A$ & $D$\\
    $1$ & $B$ & $C$
\end{tabular}
\end{table}

\vspace{1em}
\noindent
Per cui si ottiene la codifica
\[A=00 \hspace{1em} B=10 \hspace{1em} C=11 \hspace{1em} D=01\]

\noindent
\section{Esercizio}
Si semplifichi impiegando  il metodo di Ginsburg la seguente macchina sincrona:

\vspace{1em}
\noindent
\begin{table}[H]
    \rowcolors{1}{white}{white}
\setlength{\tabcolsep}{4pt}
\renewcommand{\arraystretch}{1.2}
\centering
\begin{tabular}{c|c|c|c}
    \textbf{Ingressi} & $\bf{I_1}$ & $\bf{I_2}$ & $\bf{I_3}$\\
    \textbf{Stati}    &            &            &\\
    \hline
    $A$ & $D/-$ & $A/0$ & $D/0$\\
    $B$ & $G/1$ & $C/-$ & $B/0$\\
    $C$ & $B/-$ & $C/0$ & $B/0$\\
    $D$ & $D/0$ & $A/-$ & $F/1$\\
    $E$ & $D/-$ & $-/-$ & $F/1$\\
    $F$ & $B/-$ & $C/-$ & $-/0$\\
    $G$ & $G/-$ & $C/-$ & $-/0$\\
\end{tabular}
\end{table}

\vspace{2em}
\noindent
Individuando una prima compatibilità tra le uscite si ottiene la tabella di semplificazione intermedia seguente:

\vspace{1em}
\noindent
\begin{table}[H]
    \rowcolors{1}{white}{white}
\setlength{\tabcolsep}{4pt}
\renewcommand{\arraystretch}{1.2}
\centering
\begin{tabular}{c|c|c|c}
    \textbf{Ingressi} & $\bf{I_1}$ & $\bf{I_2}$ & $\bf{I_3}$\\
    \textbf{Stati}    &            &            &\\
    \hline
    $AC$ & $DB$ & $AC$ & $DB$\\
    $AF$ & $DB$ & $AC$ & $D-$\\
    $BE$ & $GE$ & $C-$ & $AF$\\
    $BG$ & $G$  & $C$  & $A-$\\
    $CF$ & $B$  & $C$  & $B-$\\
    $DE$ & $DE$ & $A-$ & $F$\\
    $DG$ & $DG$ & $AC$ & $F-$\\
    $EG$ & $EG$ & $C-$ & $F-$\\
\end{tabular}
\end{table}

\vspace{1em}
\noindent
Tuttavia, è evidente che non tutte queste $\alpha$-compatibilità risultano essere compatibilità complete; infatti, le coppie evidenziate in seguito non sono compatibili

\vspace{1em}
\noindent
\begin{table}[H]
    \rowcolors{1}{white}{white}
\setlength{\tabcolsep}{4pt}
\renewcommand{\arraystretch}{1.2}
\centering
\begin{tabular}{c|c|c|c}
    \textbf{Ingressi} & $\bf{I_1}$ & $\bf{I_2}$ & $\bf{I_3}$\\
    \textbf{Stati}    &            &            &\\
    \hline
    \cellcolor{orange!25} $AC$ & \cellcolor{orange!25} $DB$ & \cellcolor{orange!25} $AC$ & \cellcolor{orange!25} $DB$\\
    \cellcolor{orange!25} $AF$ & \cellcolor{orange!25} $DB$ & \cellcolor{orange!25} $AC$ & \cellcolor{orange!25} $D-$\\
    \cellcolor{orange!25} $BE$ & \cellcolor{orange!25} $GE$ & \cellcolor{orange!25} $C-$ & \cellcolor{orange!25} $AF$\\
    $BG$ & $G$  & $C$  & $A-$\\
    $CF$ & $B$  & $C$  & $B-$\\
    $DE$ & $DE$ & $A-$ & $F$\\
    \cellcolor{orange!25}$DG$ & \cellcolor{orange!25} $DG$ & \cellcolor{orange!25} $AC$ & \cellcolor{orange!25} $F-$\\
    $EG$ & $EG$ & $C-$ & $F-$\\
\end{tabular}
\end{table}

\vspace{1em}
\noindent
in quanto $D$ e $B$ non sono compatibili rispetto alle uscite. Di conseguenza non possono essere compatibili $A$ e $C$ e $A$ e $F$; da questo segue l'incompatibilità anche di $B$ ed $E$, così come di $B$ e $G$.\\
A questo punto risulta abbastanza evidente che gli accorpamenti che minimizzano il numero di stati finali ed al contempo garantiscano tutte le mutue relazioni possono essere
\[S_1 = \{A\} \hspace{2em} S_2 = \{BG\} \hspace{2em} S_3 = \{CF\} \hspace{2em} S_4 = \{ED\}\]
Che portano quindi il numero degli stati finali a $4$ ed alla macchina semplificata nella seguente forma:

\vspace{1em}
\noindent
\begin{table}[H]
    \rowcolors{1}{white}{white}
\setlength{\tabcolsep}{4pt}
\renewcommand{\arraystretch}{1.2}
\centering
\begin{tabular}{c|c|c|c}
    \textbf{Ingressi} & $\bf{I_1}$ & $\bf{I_2}$ & $\bf{I_3}$\\
    \textbf{Stati}    &            &            &\\
    \hline
    $A$  & $ED/-$ & $A/0$  & $ED/0$\\
    $BG$ & $BG/1$ & $CF/-$ & $BG/0$\\
    $CF$ & $BG/-$ & $CF/0$ & $BG/0$\\
    $ED$ & $ED/0$ & $A/-$  & $CF/1$\\
\end{tabular}
\end{table}

\vspace{1em}
\noindent
oppure, rinominando gli stati:

\vspace{1em}
\noindent
\begin{table}[H]
    \rowcolors{1}{white}{white}
\setlength{\tabcolsep}{4pt}
\renewcommand{\arraystretch}{1.2}
\centering
\begin{tabular}{c|c|c|c}
    \textbf{Ingressi} & $\bf{I_1}$ & $\bf{I_2}$ & $\bf{I_3}$\\
    \textbf{Stati}    &            &            &\\
    \hline
    $1$  & $4/-$ & $1/0$ & $4/0$\\
    $2$  & $2/1$ & $3/-$ & $1/0$\\
    $3$  & $2/-$ & $3/0$ & $2/0$\\
    $4$  & $4/0$ & $1/-$ & $3/1$\\
\end{tabular}
\end{table}

\noindent
\section{Esercizio}
Descrivere attraverso la tavola di Huffman un contatore sincrono circolare a $3$ bit dotato di un bit di controllo $C$. Attraverso tale controllo si può passare dal conteggio in forma binaria al conteggio secondo il codice di Gray.

\vspace{2em}
\noindent
Si assuma per ipotesi che il controllo $C$, quando è alto, permette di contare in binario; se è basso il conteggio avverrà secondo il codice di Gray. La tavola di Huffman che descrive il funzionamento del circuito di cui sopra è

\vspace{1em}
\noindent
\begin{table}[H]
\rowcolors{1}{white}{white}
\setlength{\tabcolsep}{4pt}
\renewcommand{\arraystretch}{1.2}
\centering
\begin{tabular}{c|c|c|c}
    $C$ & $\bf{0}$ & $\bf{1}$ & $Z$\\
    \textbf{Stati} &    &     &\\
    \hline
    $A$  & $B$ & $B$ & $000$\\
    $B$  & $D$ & $C$ & $001$\\
    $C$  & $G$ & $D$ & $010$\\
    $D$  & $C$ & $E$ & $011$\\
    $E$  & $A$ & $F$ & $100$\\
    $F$  & $E$ & $G$ & $101$\\
    $G$  & $H$ & $H$ & $110$\\
    $H$  & $F$ & $A$ & $111$\\
\end{tabular}
\end{table}

\vspace{1em}
\noindent
Per comodità, si sceglie di codificare gli stati come le loro rispettive uscite, ottenendo

\vspace{1em}
\noindent
\begin{table}[H]
\rowcolors{1}{white}{white}
\setlength{\tabcolsep}{4pt}
\renewcommand{\arraystretch}{1.2}
\centering
\begin{tabular}{c|c|c|c}
    $C$ & $\bf{0}$ & $\bf{1}$ & $Z$\\
    \textbf{Stati} &    &     &\\
    \hline
    $000$  & $001$ & $001$ & $000$\\
    $001$  & $011$ & $010$ & $001$\\
    $010$  & $110$ & $011$ & $010$\\
    $011$  & $010$ & $100$ & $011$\\
    $100$  & $000$ & $101$ & $100$\\
    $101$  & $100$ & $110$ & $101$\\
    $110$  & $111$ & $111$ & $110$\\
    $111$  & $101$ & $000$ & $111$\\
\end{tabular}
\end{table}

\noindent
\section{Esercizio}
Si analizzi il seguente circuito e se ne descriva il funzionamento attraverso un grafo di MOORE:

\begin{figure}[H]
    \centering

\tikzset{every picture/.style={line width=0.75pt}} %set default line width to 0.75pt        
\begin{tikzpicture}[x=0.75pt,y=0.75pt,yscale=-1,xscale=1]
    %Shape: Nand Gate [id:dp30291443820003994] 
    \draw   (114.85,126) -- (137.13,126) .. controls (149.43,126) and (159.41,137.47) .. (159.41,151.6) .. controls (159.41,165.73) and (149.43,177.2) .. (137.13,177.2) -- (114.85,177.2) -- (114.85,126) -- cycle (100,134.53) -- (114.85,134.53) (100,168.67) -- (114.85,168.67) (168.32,151.6) -- (180.2,151.6) (159.41,151.6) .. controls (159.41,148.77) and (161.4,146.48) .. (163.86,146.48) .. controls (166.32,146.48) and (168.32,148.77) .. (168.32,151.6) .. controls (168.32,154.43) and (166.32,156.72) .. (163.86,156.72) .. controls (161.4,156.72) and (159.41,154.43) .. (159.41,151.6) -- cycle ;
    %Shape: Nand Gate [id:dp7047136364982123] 
    \draw   (257.85,110) -- (280.13,110) .. controls (292.43,110) and (302.41,121.47) .. (302.41,135.6) .. controls (302.41,149.73) and (292.43,161.2) .. (280.13,161.2) -- (257.85,161.2) -- (257.85,110) -- cycle (243,118.53) -- (257.85,118.53) (243,152.67) -- (257.85,152.67) (311.32,135.6) -- (323.2,135.6) (302.41,135.6) .. controls (302.41,132.77) and (304.4,130.48) .. (306.86,130.48) .. controls (309.32,130.48) and (311.32,132.77) .. (311.32,135.6) .. controls (311.32,138.43) and (309.32,140.72) .. (306.86,140.72) .. controls (304.4,140.72) and (302.41,138.43) .. (302.41,135.6) -- cycle ;
    %Straight Lines [id:da3706463559988238] 
    \draw    (180.2,151.6) -- (243,152.67) ;
    %Straight Lines [id:da23021528863508367] 
    \draw    (323.2,135.6) -- (424.2,135.2) ;
    %Straight Lines [id:da5047255616053188] 
    \draw    (373.2,136.2) -- (374.2,200.2) ;
    \draw [shift={(373.2,136.2)}, rotate = 89.1] [color={rgb, 255:red, 0; green, 0; blue, 0 }  ][fill={rgb, 255:red, 0; green, 0; blue, 0 }  ][line width=0.75]      (0, 0) circle [x radius= 3.35, y radius= 3.35]   ;
    %Straight Lines [id:da12592223348600973] 
    \draw    (59.2,198.2) -- (374.2,200.2) ;
    %Straight Lines [id:da8489578409787053] 
    \draw    (100,168.67) -- (59.2,168.2) ;
    %Straight Lines [id:da9144659333978256] 
    \draw    (59.2,168.2) -- (59.2,198.2) ;
    %Straight Lines [id:da18588210914724035] 
    \draw    (198.2,80.2) -- (199.2,152.2) ;
    \draw [shift={(199.2,152.2)}, rotate = 89.2] [color={rgb, 255:red, 0; green, 0; blue, 0 }  ][fill={rgb, 255:red, 0; green, 0; blue, 0 }  ][line width=0.75]      (0, 0) circle [x radius= 3.35, y radius= 3.35]   ;

    % Text Node
    \draw (436,126.4) node [anchor=north west][inner sep=0.75pt]    {$Z_{2}$};
    % Text Node
    \draw (188,56.4) node [anchor=north west][inner sep=0.75pt]    {$Z_{1}$};
    % Text Node
    \draw (81.6,171.83) node [anchor=north west][inner sep=0.75pt]    {$y$};
    % Text Node
    \draw (328,144.4) node [anchor=north west][inner sep=0.75pt]    {$y'$};
    % Text Node
    \draw (78,126.4) node [anchor=north west][inner sep=0.75pt]    {$A$};
    % Text Node
    \draw (221,110.4) node [anchor=north west][inner sep=0.75pt]    {$B$};
\end{tikzpicture}
\end{figure}

\vspace{2em}
\noindent
Denominata come $y$ la variabile di stato in ingresso alla logica stessa e con $y'$ la variabile di stato da questa generata, si possono evidenziare le funzioni logiche che forniscono  le relazioni tra ingressi ed uscite (e tra le uscite si considera anche $y'$).
\[Z_1 = \overline{A \cdot y} \hspace{2em} y'=\overline{\overline{A \cdot y} \cdot B} = A \cdot y + \overline{B} \hspace{2em} Z_2 = y'\]
Realizzando, ora, la tavole di eccitazione si ottiene

\vspace{1em}
\noindent
\begin{table}[H]
\rowcolors{1}{white}{white}
\setlength{\tabcolsep}{4pt}
\renewcommand{\arraystretch}{1.2}
\centering
\begin{tabular}{ccccc}
    {
        \begin{tabular}{c|c|c|c|c|}
            $AB$ & $00$ & $01$ & $11$ & $10$\\
            $y$  &      &      &      &\\
            \hline
            $0$  & $1$ & $1$ & $1$ & $1$\\
            \hline
            $1$  & $1$ & $1$ & $0$ & $0$\\
            \hline
            \multicolumn{4}{c}{$Z_1$}\\
        \end{tabular}
    }&&
    {
        \begin{tabular}{c|c|c|c|c|}
            $AB$ & $00$ & $01$ & $11$ & $10$\\
            $y$  &      &      &      &\\
            \hline
            $0$  & $1$ & $0$ & $0$ & $1$\\
            \hline
            $1$  & $1$ & $0$ & $1$ & $1$\\
            \hline
            \multicolumn{4}{c}{$y$}\\
        \end{tabular}
    }&&
    {
        \begin{tabular}{c|c|c|c|c|}
            $AB$ & $00$ & $01$ & $11$ & $10$\\
            $y$  &      &      &      &\\
            \hline
            $0$  & $1$ & $0$ & $0$ & $1$\\
            \hline
            $1$  & $1$ & $0$ & $1$ & $1$\\
            \hline
            \multicolumn{4}{c}{$Z_2$}\\
        \end{tabular}
    }\\
\end{tabular}
\end{table}

\vspace{1em}
\noindent
Tali tabelle, raccolte congiuntamente in una sola tabella forniscono la una tavola di flusso del dispositivo, ove si si trova, per ogni stato del circuito e per ogni combinazione di ingressi quale sarà lo stato futuro e le uscite corrispondenti.

\vspace{1em}
\noindent
\begin{table}[H]
\rowcolors{1}{white}{white}
\setlength{\tabcolsep}{4pt}
\renewcommand{\arraystretch}{1.2}
\centering
    \begin{tabular}{c|c|c|c|c|}
        $AB$ & $00$ & $01$ & $11$ & $10$\\
        $y$  &      &      &      &\\
        \hline
        $0$  & $1/11$ & \cellcolor{orange!25!white}$0/10$ & \cellcolor{orange!25!white}$0/10$ & $1/11$\\
        \hline
        $1$  & \cellcolor{orange!25!white}$1/11$ & $0/10$ & \cellcolor{orange!25!white}$1/01$ & \cellcolor{orange!25!white}$1/01$\\
    \end{tabular}
\end{table}

\vspace{1em}
\noindent
Da ciò segue il grafico di Moore seguente

\begin{figure}[H]
    \centering
\tikzset{every picture/.style={line width=0.75pt}} %set default line width to 0.75pt        

\begin{tikzpicture}[x=0.75pt,y=0.75pt,yscale=-1,xscale=1]
%uncomment if require: \path (0,300); %set diagram left start at 0, and has height of 300

%Shape: Circle [id:dp6236917376988486] 
\draw   (135,192) .. controls (135,178.19) and (146.19,167) .. (160,167) .. controls (173.81,167) and (185,178.19) .. (185,192) .. controls (185,205.81) and (173.81,217) .. (160,217) .. controls (146.19,217) and (135,205.81) .. (135,192) -- cycle ;
%Shape: Circle [id:dp37731619104585323] 
\draw   (268,86) .. controls (268,72.19) and (279.19,61) .. (293,61) .. controls (306.81,61) and (318,72.19) .. (318,86) .. controls (318,99.81) and (306.81,111) .. (293,111) .. controls (279.19,111) and (268,99.81) .. (268,86) -- cycle ;
%Shape: Circle [id:dp2253603683782387] 
\draw   (411,195) .. controls (411,181.19) and (422.19,170) .. (436,170) .. controls (449.81,170) and (461,181.19) .. (461,195) .. controls (461,208.81) and (449.81,220) .. (436,220) .. controls (422.19,220) and (411,208.81) .. (411,195) -- cycle ;
%Curve Lines [id:da7912368963830844] 
\draw    (135,192) .. controls (89.66,207.64) and (85.28,142.13) .. (137.6,174.78) ;
\draw [shift={(139.2,175.8)}, rotate = 212.95] [color={rgb, 255:red, 0; green, 0; blue, 0 }  ][line width=0.75]    (10.93,-3.29) .. controls (6.95,-1.4) and (3.31,-0.3) .. (0,0) .. controls (3.31,0.3) and (6.95,1.4) .. (10.93,3.29)   ;
%Curve Lines [id:da12286695252978763] 
\draw    (153.2,216.8) .. controls (136.37,266.3) and (91.12,225.63) .. (136.79,206.38) ;
\draw [shift={(138.2,205.8)}, rotate = 158.4] [color={rgb, 255:red, 0; green, 0; blue, 0 }  ][line width=0.75]    (10.93,-3.29) .. controls (6.95,-1.4) and (3.31,-0.3) .. (0,0) .. controls (3.31,0.3) and (6.95,1.4) .. (10.93,3.29)   ;
%Curve Lines [id:da2816777953980044] 
\draw    (160,167) .. controls (160,131.36) and (214.1,85.72) .. (266.42,85.98) ;
\draw [shift={(268,86)}, rotate = 181.3] [color={rgb, 255:red, 0; green, 0; blue, 0 }  ][line width=0.75]    (10.93,-3.29) .. controls (6.95,-1.4) and (3.31,-0.3) .. (0,0) .. controls (3.31,0.3) and (6.95,1.4) .. (10.93,3.29)   ;
%Curve Lines [id:da1929233284222901] 
\draw    (318,86) .. controls (373.36,72.8) and (432.4,126.94) .. (435.88,168.13) ;
\draw [shift={(436,170)}, rotate = 267.51] [color={rgb, 255:red, 0; green, 0; blue, 0 }  ][line width=0.75]    (10.93,-3.29) .. controls (6.95,-1.4) and (3.31,-0.3) .. (0,0) .. controls (3.31,0.3) and (6.95,1.4) .. (10.93,3.29)   ;
%Curve Lines [id:da13946859618987417] 
\draw    (281.2,62.6) .. controls (271.35,17.29) and (322.62,29.22) .. (307.92,62.08) ;
\draw [shift={(307.2,63.6)}, rotate = 296.57] [color={rgb, 255:red, 0; green, 0; blue, 0 }  ][line width=0.75]    (10.93,-3.29) .. controls (6.95,-1.4) and (3.31,-0.3) .. (0,0) .. controls (3.31,0.3) and (6.95,1.4) .. (10.93,3.29)   ;
%Curve Lines [id:da3041301354770417] 
\draw    (452.2,174.6) .. controls (470.02,112.23) and (526.06,196.87) .. (463.14,184.99) ;
\draw [shift={(461.2,184.6)}, rotate = 11.98] [color={rgb, 255:red, 0; green, 0; blue, 0 }  ][line width=0.75]    (10.93,-3.29) .. controls (6.95,-1.4) and (3.31,-0.3) .. (0,0) .. controls (3.31,0.3) and (6.95,1.4) .. (10.93,3.29)   ;
%Curve Lines [id:da9654589650830687] 
\draw    (460.2,205.6) .. controls (498.61,238.1) and (437.1,263.82) .. (436.01,221.96) ;
\draw [shift={(436,220)}, rotate = 91.03] [color={rgb, 255:red, 0; green, 0; blue, 0 }  ][line width=0.75]    (10.93,-3.29) .. controls (6.95,-1.4) and (3.31,-0.3) .. (0,0) .. controls (3.31,0.3) and (6.95,1.4) .. (10.93,3.29)   ;
%Curve Lines [id:da9597962897167698] 
\draw    (419.2,217.4) .. controls (401.38,243.14) and (224.78,246.53) .. (175.65,216.52) ;
\draw [shift={(174.2,215.6)}, rotate = 33.41] [color={rgb, 255:red, 0; green, 0; blue, 0 }  ][line width=0.75]    (10.93,-3.29) .. controls (6.95,-1.4) and (3.31,-0.3) .. (0,0) .. controls (3.31,0.3) and (6.95,1.4) .. (10.93,3.29)   ;
%Curve Lines [id:da4815762441016117] 
\draw    (280.2,108.4) .. controls (270.3,146.02) and (235.9,196.58) .. (186.5,192.15) ;
\draw [shift={(185,192)}, rotate = 6.37] [color={rgb, 255:red, 0; green, 0; blue, 0 }  ][line width=0.75]    (10.93,-3.29) .. controls (6.95,-1.4) and (3.31,-0.3) .. (0,0) .. controls (3.31,0.3) and (6.95,1.4) .. (10.93,3.29)   ;
%Curve Lines [id:da26262578473342046] 
\draw    (411,195) .. controls (384.6,197.36) and (320.17,148.89) .. (302.76,111.69) ;
\draw [shift={(302,110)}, rotate = 66.58] [color={rgb, 255:red, 0; green, 0; blue, 0 }  ][line width=0.75]    (10.93,-3.29) .. controls (6.95,-1.4) and (3.31,-0.3) .. (0,0) .. controls (3.31,0.3) and (6.95,1.4) .. (10.93,3.29)   ;

% Text Node
\draw (141,184.4) node [anchor=north west][inner sep=0.75pt]    {$0/10$};
% Text Node
\draw (274,78.4) node [anchor=north west][inner sep=0.75pt]    {$1/11$};
% Text Node
\draw (417,187.4) node [anchor=north west][inner sep=0.75pt]    {$1/01$};
% Text Node
\draw (103,240.4) node [anchor=north west][inner sep=0.75pt]    {$01$};
% Text Node
\draw (76,165.4) node [anchor=north west][inner sep=0.75pt]    {$11$};
% Text Node
\draw (173,95.4) node [anchor=north west][inner sep=0.75pt]    {$00$};
% Text Node
\draw (193,115.4) node [anchor=north west][inner sep=0.75pt]    {$10$};
% Text Node
\draw (411,100.4) node [anchor=north west][inner sep=0.75pt]    {$10$};
% Text Node
\draw (285,14.4) node [anchor=north west][inner sep=0.75pt]    {$00$};
% Text Node
\draw (464,246.4) node [anchor=north west][inner sep=0.75pt]    {$10$};
% Text Node
\draw (496,149.4) node [anchor=north west][inner sep=0.75pt]    {$11$};
% Text Node
\draw (299,242.4) node [anchor=north west][inner sep=0.75pt]    {$01$};
% Text Node
\draw (258,161.4) node [anchor=north west][inner sep=0.75pt]    {$01$};
% Text Node
\draw (322,161.4) node [anchor=north west][inner sep=0.75pt]    {$00$};
\end{tikzpicture}
\end{figure}



















\noindent
\section{Dichiarazione}
Il lavoro di cui sopra è stato svolto da me in completa autonomia.
\end{document}